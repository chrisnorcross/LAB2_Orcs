\documentclass{article}

\usepackage{times}

\begin{document}

\subsection*{
CISC 275 Lab: Animation }

To receive credit for today's lab you must work with a {\bf different person/people}
than you worked with in the first lab. Remember that part of the purpose of these assignments is to have you meet and work with other people so that you will be able to form good teams later.

Getting this code to work is simple. Adding parts in a nice way requires thinking: \emph{software engineering}. Try to think in terms of writing so that your code would be easy for someone else to modify/maintain.

\subsection*{Non-graded portions}
\label{sec:non-graded-portions}

Get Animation.java running.You must place the orc images in the correct directory structure for the file (see code). You may work with others to get it working.

Ultimately you will need the other orc images (provided in the .zip). 

\subsection*{Graded Exercise – On your own or in new pairs
}
\label{sec:graded-exercise-}

Modify Animation.java:

\begin{itemize}

\item See the three code sections marked TODO. Read the comments.

\item Keep orc from walking off screen

\item Make orc bounce off the edges. The orc will then travel in a new
  direction and you will need to use different image sequences for the
  new direction. Use at least four different directions. Only read
  each .png file once (why?).
\end{itemize}

Hint: use flags for changing directions

Hint: Avoid hard-coding any numerical values (why?)

See the TODO notes in the source file for suggestions. Submit your Eclipse project exported as .zip to Canvas\footnote{A plain zipped directory is not sufficient, it must be an Eclipse export.}.


\end{document}

